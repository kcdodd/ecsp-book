\chapter{Source Code}


Source code is text that is readable by a human, but that can be translated into operations that a computer knows how to perform. Source code is always written in a plain text file, and can be edited by any plain text editor.

\section{Programming Languages}

A programming language is one particular way of representing and organizing a program at the human readable level in source code. It is an abstraction of the underlying operations that the computer will perform when the program is executed. In some sense, a programming language is arbitrary since there is usually more than one way to think about or describe the same thing. But since we require a programming language to be able to represent any program, there is a minimum set of concepts it must be capable of expressing.\\

Before a program can be executed, the source code must be translated into those operations the computer understands. Fortunately you don't have to do that step. The idea of a programming language is that it is specific enough that another program can be written to do the translation for you. That program is either called an interpretor, or a compiler, depending on when and how the translation step is done.


\section{JavaScript}

Although this book focuses on learning the JavaScript language, the larger idea is to understand the underlying concepts because they appear in other languages. Instead of thinking of yourself as becoming a JavaScript programmer, you should try to think of yourself as becoming a general programmer. Once you learn one language, and understand the underlying ideas, learning another language should be easier because you know what to look for.\\

Just as different programming languages express the same concepts in different ways, JavaScript itself contains different ways to express the same concepts. This is because of how JavaScript has evolved since its creation. Programs can be written entirely using a subset of the language. You might even consider different sub-sets as dialects of JavaScript because the programs will look different, even though they are all written in the same overall language.\\

This book is not meant as a comprehensive course on JavaScript and all the different features it contains. We will focus on one particular sub-set of the language and a way of writing programs. If you decide to use code written by other programmers, just remember that they may not write their programs the same way we have.

\section{HTML and CSS}

HTML stands for HyperText Markup Language. There is some debate as to whether HTML qualifies as a programming language. What we mean by programming language has to do with what it is able to compute, or what algorithms it is able to implement. HTML cannot be used to compute things in an obvious way, so for our purposes we may not consider it as a programming language.\\

However, we do use it to specify information in a web browser in a human readable fashion, such as what files to load and how to display things, which the browser then interprets into the operations it needs to accomplish those tasks. Often accompanying HTML is CSS, which stands for Cascading Style Sheets. CSS is an additional notation used to help specify the display of content in the browser.

\section{Style and Conventions}

\section{Organization and Version Control}