\chapter{Logic and Binary Arithmetic}

The logical operators (AND, OR, NOT) are ways of combining truth values, resulting in a new truth value. You can think of them kind of like arithmetic operations, except for boolean values (true or false) instead of numbers.\\

\section{The OR Operator}

To illustrate how they work, imagine someone is trying to get into a locked room full of gold. The room has some set of doors that leads to the room, and each door requires a key to open. In this first scenario, the room has two doors either one of which will let the person in. If they have at least one of the keys, they can get in. This is like the OR operation.\\

\begin{center} \imagegraphic[1.0]{OR.png}\end{center}

Now to put this in more mathematical terms, the truth values of A and B are referring to whether that person has the key for that door. And the resulting truth value is whether they are able to get the gold. So, if I say \(A=true\), then I am saying they have the key for door A. \(A=false\) would mean that they don't have that key. And \(B=true\) means they have the key for door B, etc.\\

When I say \(A\: {or} \: B\), the result is true if they have either key, even if they have both keys at the same time. The result is false only if they have neither key. There are two keys, and there are two truth values for each one (do they have it or not). \(2 \times 2 = 4\) possible combinations. If it helps you, every possible result can be listed in a table\\

\begin{center}
	\begin{tabular}{c | c | c}
		has key A & has key B & A or B = gets the gold\\ \hline
		\textcolor{red}{false} & \textcolor{red}{false} & \textcolor{red}{false}\\ \hline
		\textcolor{blue}{true} & \textcolor{red}{false} & \textcolor{blue}{true} \\ \hline
		\textcolor{red}{false} & \textcolor{blue}{true} & \textcolor{blue}{true} \\ \hline
		\textcolor{blue}{true} & \textcolor{blue}{true} &\textcolor{blue} {true} \\ \hline
	\end{tabular}
\end{center}

\section{The AND Operator}

In the next scenario, the one door is behind the other door. If the person has key A, but not key B, they cannot get in. If they have key B but not key A, they still cannot get in. They have to have both keys to get through.\\

\begin{center} \imagegraphic[1.0]{AND.png}\end{center}

\begin{center}
	\begin{tabular}{c | c | c}
		has key A & has key B & A and B = gets the gold \\ \hline
		\textcolor{red}{false} & \textcolor{red}{false} & \textcolor{red}{false}\\ \hline
		\textcolor{blue}{true} & \textcolor{red}{false} & \textcolor{red}{false} \\ \hline
		\textcolor{red}{false} & \textcolor{blue}{true} & \textcolor{red}{false} \\ \hline
		\textcolor{blue}{true} & \textcolor{blue}{true} & \textcolor{blue}{true} \\ \hline
	\end{tabular}
\end{center}

\section{Order of Operations}

Logical operators can be combined just like arithmetic operators can. Parenthesis can be used to specify the order of the operations. In this example, the operation A or B is performed first. The result of that is then used with the AND operation. To get into the first room, they only need one key: either A or B. But to get into the gold room, they also need key C no matter what.\\

\begin{center} \imagegraphic[1.0]{AorBandC.png}\end{center}

This time there are 8 total possible outcomes because there are three keys, each with two possible states (has it or doesn't have it): \(2 \times 2 \times 2 = 8 \). But since they can only get in if they also have key C, then only those times can they get the gold.

\begin{center}
	\begin{tabular}{c | c | c | c | c}
		A & B & C & A or B & (A or B) and C = gets the gold\\ \hline
		\textcolor{red}{false} & \textcolor{red}{false} & \textcolor{red}{false} & \textcolor{red}{false} & \textcolor{red}{false}\\ \hline
		\textcolor{red}{false} & \textcolor{blue}{true} & \textcolor{red}{false} & \textcolor{blue}{true} & \textcolor{red}{false}\\ \hline
		\textcolor{blue}{true} & \textcolor{red}{false} & \textcolor{red}{false} & \textcolor{blue}{true} & \textcolor{red}{false}\\ \hline
		\textcolor{blue}{true} & \textcolor{blue}{true} & \textcolor{red}{false} & \textcolor{blue}{true} & \textcolor{red}{false}\\ \hline

		\textcolor{red}{false} & \textcolor{red}{false} & \textcolor{blue}{true} & \textcolor{red}{false} & \textcolor{red}{false}\\ \hline
		\textcolor{red}{false} & \textcolor{blue}{true} & \textcolor{blue}{true} & \textcolor{blue}{true} & \textcolor{blue}{true}\\ \hline
		\textcolor{blue}{true} & \textcolor{red}{false} & \textcolor{blue}{true} & \textcolor{blue}{true} & \textcolor{blue}{true}\\ \hline
		\textcolor{blue}{true} & \textcolor{blue}{true} & \textcolor{blue}{true} & \textcolor{blue}{true} & \textcolor{blue}{true}\\ \hline
	\end{tabular}
\end{center}

Logical operations can usually be thought of in different ways. We could have also thought about the above problem as having two different ways of getting into the gold room. Either having keys A and C, or having keys B and C. So, we could write the same result as: (A and C) or (B and C). But it means the same thing.

If the AND operation is done first, however, that means that even if they don't have either key B or key C, they can still get in if they have key A. But if they don't have key A, then they have to have both keys B and C to get in.\\

\begin{center} \imagegraphic[1.0]{AorBandC2.png}\end{center}

\begin{center}
	\begin{tabular}{c | c | c | c | c}
		A & B & C & B and C & A or (B and C) = gets the gold\\ \hline
		\textcolor{red}{false} & \textcolor{red}{false} & \textcolor{red}{false} & \textcolor{red}{false} & \textcolor{red}{false}\\ \hline
		\textcolor{red}{false} & \textcolor{blue}{true} & \textcolor{red}{false} & \textcolor{red}{false} & \textcolor{red}{false}\\ \hline
		\textcolor{blue}{true} & \textcolor{red}{false} & \textcolor{red}{false} & \textcolor{red}{false} & \textcolor{blue}{true}\\ \hline
		\textcolor{blue}{true} & \textcolor{blue}{true} & \textcolor{red}{false} & \textcolor{red}{false} & \textcolor{blue}{true}\\ \hline

		\textcolor{red}{false} & \textcolor{red}{false} & \textcolor{blue}{true} & \textcolor{red}{false} & \textcolor{red}{false}\\ \hline
		\textcolor{red}{false} & \textcolor{blue}{true} & \textcolor{blue}{true} & \textcolor{blue}{true} & \textcolor{blue}{true}\\ \hline
		\textcolor{blue}{true} & \textcolor{red}{false} & \textcolor{blue}{true} & \textcolor{red}{false} & \textcolor{blue}{true}\\ \hline
		\textcolor{blue}{true} & \textcolor{blue}{true} & \textcolor{blue}{true} & \textcolor{blue}{true} & \textcolor{blue}{true}\\ \hline
	\end{tabular}
\end{center}


If there are no parenthesis, there are a set of rules for the order of operations. Luckily there is only 1 rule: do the AND first. So, A or B and C = A or (B and C), since AND gets higher precedence anyway. However, always using parenthesis has the added benefit of always making you think about want you \textit{intend} to do, which helps prevent mistakes.

\section{The NOT Operator}

