\chapter{Binary Logic}

The logical operators (AND, OR, NOT) are ways of combining truth values, resulting in a new truth value. You can think of them kind of like arithmetic operations, except for boolean values (true or false) instead of numbers.\\

\section{The OR Operator}

To illustrate how they work, imagine someone is trying to get into a locked room full of gold. The room has some set of doors that leads to the room, and each door requires a key to open. In this first scenario, the room has two doors either one of which will let the person in. If they have at least one of the keys, they can get in. This is like the OR operation.\\

\begin{center} \imagegraphic[1.0]{OR.png}\end{center}

Now to put this in more mathematical terms, the truth values of A and B are referring to whether that person has the key for that door. And the resulting truth value is whether they are able to get the gold. So, if I say \(A=true\), then I am saying they have the key for door A. \(A=false\) would mean that they don't have that key. And \(B=true\) means they have the key for door B, etc.\\

When I say \(A\: OR \: B\), the result is true if they have either key, even if they have both keys at the same time. The result is false only if they have neither key. There are two keys, and there are two truth values for each one (do they have it or not). \(2 \times 2 = 4\) possible combinations. If it helps you, every possible result can be listed in a table\\

\begin{center}
	\begin{tabular}{c | c | c}
		has key A & has key B & A OR B = gets the gold\\ \hline
		\textcolor{red}{false} & \textcolor{red}{false} & \textcolor{red}{false}\\ \hline
		\textcolor{blue}{true} & \textcolor{red}{false} & \textcolor{blue}{true} \\ \hline
		\textcolor{red}{false} & \textcolor{blue}{true} & \textcolor{blue}{true} \\ \hline
		\textcolor{blue}{true} & \textcolor{blue}{true} &\textcolor{blue} {true} \\ \hline
	\end{tabular}
\end{center}

We can also view the OR operations as the result of a flowchart. You can see below why there is only one way of getting a false result with the OR operation: both A and B have to be false to get false. But there is more than one way to get true.

\begin{center} \imagegraphic[0.5]{flowchart_OR.png}\end{center}

\section{The AND Operator}

In the next scenario, the one door is behind the other door. If the person has key A, but not key B, they cannot get in. If they have key B but not key A, they still cannot get in. \\

\begin{center} \imagegraphic[1.0]{AND.png}\end{center}

So, the result of A AND B is only true if they have both keys; that is, both A and B are true. If either A or B are false, or both, then the result is false.

\begin{center}
	\begin{tabular}{c | c | c}
		has key A & has key B & A AND B = gets the gold \\ \hline
		\textcolor{red}{false} & \textcolor{red}{false} & \textcolor{red}{false}\\ \hline
		\textcolor{blue}{true} & \textcolor{red}{false} & \textcolor{red}{false} \\ \hline
		\textcolor{red}{false} & \textcolor{blue}{true} & \textcolor{red}{false} \\ \hline
		\textcolor{blue}{true} & \textcolor{blue}{true} & \textcolor{blue}{true} \\ \hline
	\end{tabular}
\end{center}

We can also view the AND operation as the result of a flowchart, and why there is only one way of getting a true result from an AND operation; both A and B have to be true. But there is more than one way to get false.

\begin{center} \imagegraphic[0.3]{flowchart_AND.png}\end{center}

\section{Order of Operations}

Logical operators can be combined just like arithmetic operators can. Parenthesis can be used to specify the order of the operations. In this example, the operation A or B is performed first. The result of that is then used with the AND operation. To get into the first room, they only need one key: either A OR B. But to get into the gold room, they also need key C no matter what.\\

\begin{center} \imagegraphic[1.0]{AorBandC.png}\end{center}

This time there are 8 total possible outcomes because there are three keys, each with two possible states (has it or doesn't have it): \(2 \times 2 \times 2 = 8 \). But since they can only get in if they also have key C, then only those times can they get the gold.

\begin{center}
	\begin{tabular}{c | c | c | c | c}
		A & B & C & A OR B & (A OR B) AND C = gets the gold\\ \hline
		\textcolor{red}{false} & \textcolor{red}{false} & \textcolor{red}{false} & \textcolor{red}{false} & \textcolor{red}{false}\\ \hline
		\textcolor{red}{false} & \textcolor{blue}{true} & \textcolor{red}{false} & \textcolor{blue}{true} & \textcolor{red}{false}\\ \hline
		\textcolor{blue}{true} & \textcolor{red}{false} & \textcolor{red}{false} & \textcolor{blue}{true} & \textcolor{red}{false}\\ \hline
		\textcolor{blue}{true} & \textcolor{blue}{true} & \textcolor{red}{false} & \textcolor{blue}{true} & \textcolor{red}{false}\\ \hline

		\textcolor{red}{false} & \textcolor{red}{false} & \textcolor{blue}{true} & \textcolor{red}{false} & \textcolor{red}{false}\\ \hline
		\textcolor{red}{false} & \textcolor{blue}{true} & \textcolor{blue}{true} & \textcolor{blue}{true} & \textcolor{blue}{true}\\ \hline
		\textcolor{blue}{true} & \textcolor{red}{false} & \textcolor{blue}{true} & \textcolor{blue}{true} & \textcolor{blue}{true}\\ \hline
		\textcolor{blue}{true} & \textcolor{blue}{true} & \textcolor{blue}{true} & \textcolor{blue}{true} & \textcolor{blue}{true}\\ \hline
	\end{tabular}
\end{center}

As a flowchart, I simply added an AND using the result of the OR operation.

\begin{center} \imagegraphic[0.5]{flowchart_ORAND.png}\end{center}

Logical operations can usually be thought of in different ways. We could have also thought about the above problem as having two different ways of getting into the gold room. Either having keys A AND C, or having keys B AND C. So, we could write the same result as:\\

\begin{center}
	(A AND C) OR (B AND C) = (A OR B) AND C
\end{center}

In the following scenario the AND operation is done first, however, so that means that even if they don't have either key B or key C, they can still get in if they have key A. But if they don't have key A, then they have to have both keys B and C to get in.\\

\begin{center} \imagegraphic[1.0]{AorBandC2.png}\end{center}

\begin{center}
	\begin{tabular}{c | c | c | c | c}
		A & B & C & B AND C & A OR (B AND C) = gets the gold\\ \hline
		\textcolor{red}{false} & \textcolor{red}{false} & \textcolor{red}{false} & \textcolor{red}{false} & \textcolor{red}{false}\\ \hline
		\textcolor{red}{false} & \textcolor{blue}{true} & \textcolor{red}{false} & \textcolor{red}{false} & \textcolor{red}{false}\\ \hline
		\textcolor{blue}{true} & \textcolor{red}{false} & \textcolor{red}{false} & \textcolor{red}{false} & \textcolor{blue}{true}\\ \hline
		\textcolor{blue}{true} & \textcolor{blue}{true} & \textcolor{red}{false} & \textcolor{red}{false} & \textcolor{blue}{true}\\ \hline

		\textcolor{red}{false} & \textcolor{red}{false} & \textcolor{blue}{true} & \textcolor{red}{false} & \textcolor{red}{false}\\ \hline
		\textcolor{red}{false} & \textcolor{blue}{true} & \textcolor{blue}{true} & \textcolor{blue}{true} & \textcolor{blue}{true}\\ \hline
		\textcolor{blue}{true} & \textcolor{red}{false} & \textcolor{blue}{true} & \textcolor{red}{false} & \textcolor{blue}{true}\\ \hline
		\textcolor{blue}{true} & \textcolor{blue}{true} & \textcolor{blue}{true} & \textcolor{blue}{true} & \textcolor{blue}{true}\\ \hline
	\end{tabular}
\end{center}

Now that the AND is performed first, and OR is second, it causes a minor difference in the flowchart. See if you can spot the difference, and why that means there are more ways for it to result in true than when the OR was done first.

\begin{center} \imagegraphic[0.5]{flowchart_ANDOR.png}\end{center}

If there are no parenthesis, then AND is done first. So:

\begin{center}A OR B AND C = A OR (B AND C)\end{center}
	
since AND gets higher precedence anyway. However, always using parenthesis has the added benefit of always making you think about want you \textit{intend} to do, which helps prevent mistakes.

\section{The NOT Operator}

The NOT operation simply takes a truth value, and gives whatever the opposite value would be. For example, if A = false (they don't have key A), then that means that NOT A = true. It is just like saying "they don't have key A" (which is true if A = false). If A = true, then NOT A = false, since "they don't have key A" is a false statement in that case.\\

We can also view the NOT operation as the result of a flowchart.

\begin{center} \imagegraphic[0.3]{flowchart_NOT.png}\end{center}

This leads to several important identities. Take the first example where we have A OR B. So, lets say A OR B = G, where G is true when they can get the gold. But say we want to know when NOT G, when do they not get the gold? Well logically they don't get the gold if they don't have either key: NOT G = (NOT A) AND (NOT B). But if we replace the G with the original thing we get the following identity:

\begin{center}
	NOT (A OR B) = (NOT A) AND (NOT B)
\end{center}

In the view of the flowchart, it is as simple as flipping the results. True becomes false, and false becomes true. 

\begin{center} \imagegraphic[0.5]{flowchart_NOTOR.png}\end{center}

But then where does AND factor in on the right hand side? It's there because it's doing an AND on the 'no' branches now, instead of the 'yes'. I'll rearrange it a little by flipping the no/yes branches coming out of each condition so you can see it really does mean AND.

\begin{center} \imagegraphic[0.3]{flowchart_NOTOR2.png}\end{center}

Now lets do the same thing for the second case where G = A AND B. Logically, they won't get the gold if they don't have one of the keys. So, NOT G = (NOT A) OR (NOT B). Now replacing G with A AND B we get:

\begin{center}
	NOT (A AND B) = (NOT A) OR (NOT B)
\end{center}

\begin{center} \imagegraphic[0.3]{flowchart_NOTAND.png}\imagegraphic[0.47]{flowchart_NOTAND2.png}\end{center}

You may notice a pattern here. When you negate an OR, you negate both sides of the OR and then switch the OR to an AND. Also, when you negate an AND, you negate both sides and switch it to an OR.\\

To negate more complicated expressions, like (A OR B) AND C, you could do one of two things. One, you cold look at the problem and figure out the case where they don't get the gold. Here they can't get it if they're missing both the A key and the B key, or they're missing the C key. So:

\begin{center}
	NOT((A OR B) AND C) = ((NOT A) AND (NOT B)) OR (NOT C)
\end{center}

If you notice though, once you have the identities for AND and OR by themselves, you can apply the NOT starting from the 'outside' and working in. For the case G = A OR (B AND C), the last operation to be done is the OR. So, negate the OR first, which causes B AND C to be negated, so you can use the other identity to negate that one.

\begin{center}
	NOT(A OR (B AND C)) = (NOT A) AND (NOT (B AND C))\\
	
	NOT(A OR (B AND C)) = (NOT A) AND ((NOT B) OR (NOT C))
\end{center}

So, for the case of G = A OR (B AND C), we can see they don't get the gold if don't have key A, and they also don't have just one of key B or key C.\\