\chapter{Information}

\epigraph{The information of a message can be defined as the 'minimum number of binary decisions which enable the receiver to construct the message, on the basis of the data already available to him.'}{Dennis Gabor}

We use the word information frequently. Think of an example that you think contains information. Perhaps you thought of a book or a website. Can you identify what exactly about these things contain information, or how you might quantify an amount of information it contains?

\section{20 Questions}

If you are not familiar with it, 20 Questions is a game where one person thinks of an object which other people must guess. The person who knows the object may be asked any yes or no question until 20 such questions have been answered. The premise of the game is that if the questions are clever enough, only 20 questions are needed to guess what it is. But, if the questions are not good then it may not be enough to guess the object. But think of how many objects actually exist; could 20 questions really be enough to narrow it down?\\

Lets start with a simple example and limit the possibilities to the letters of the English alphabet. There are 26 letters. How many yes or no questions are required to guess one of the letters? One could adopt the strategy of asking questions like "is it the letter A?". This is a gamble. On the one hand, the letter could be guessed on the first try if the answer is 'yes'. But this type of question only eliminates one possibility at a time if the answer is 'no', and so it could take up to 25 questions to eliminate all but one of the letters.\\

\begin{verbatim}
         ABCDEFGHIJKLMNOPQRSTUVWXYZ
Guess A:  BCDEFGHIJKLMNOPQRSTUVWXYZ
Guess I:  BCDEFGH JKLMNOPQRSTUVWXYZ
Guess E:  BCD FGH JKLMNOPQRSTUVWXYZ
Guess Y:  BCD FGH JKLMNOPQRSTUVWX Z
Guess X:  BCD FGH JKLMNOPQRSTUVW  Z
Guess G:  BCD F H JKLMNOPQRSTUVW  Z
Guess Z:  BCD F H JKLMNOPQRSTUVW
Guess Q:  BCD F H JKLMNOP RSTUVW
Guess P:  BCD F H JKLMNO  RSTUVW
Guess V:  BCD F H JKLMNO  RSTU W
Guess U:  BCD F H JKLMNO  RST  W
Guess W:  BCD F H JKLMNO  RST
Guess O:  BCD F H JKLMN   RST
Guess M:  BCD F H JKL N   RST
Guess S:  BCD F H JKL N   R T
Guess T:  BCD F H JKL N   R  
Guess R:  BCD F H JKL N   
Guess B:   CD F H JKL N   
Guess D:   C  F H JKL N   
Guess K:   C  F H J L N   
Guess C:      F H J L N   
Guess J:      F H   L N
Guess L:      F H     N
Guess N:      F H
Guess F:        H
\end{verbatim}

We could ask questions like "does the letter come before the letter N?". If the answer is 'yes', then half of the letters are eliminated and half remain as possibilities. If the answer is 'no', then the other half are eliminated and half remain. The 13 remaining letters can be divided again, and again eliminating half of the letters with each 'yes' or 'no' answer until only one letter remains. So, at most 5 answers are needed to guess the letter.\\

\begin{verbatim}
           ABCDEFGHIJKLMNOPQRSTUVWXYZ
Guess < N: ABCDEFGHIJKLM
Guess < G:       GHIJKLM
Guess < J:       GHI
Guess < H:        HI
Guess < I:        H
\end{verbatim}

These relationships can be represented by what is called a binary tree. The 'root' of the tree is at the top. From each node in the tree there are two possible ways to go further down the tree. Each time the tree splits this represents a yes/no or left/right, or whatever the \textit{binary} difference may be which is used to exclude the other branches of the tree. Looking at this figure it can be seen that any letter can be guessed in either 4 or 5 guesses.

\index{binary tree}
\begin{center}\imagegraphic[0.75]{letters.png}\end{center}

This kind of strategy seems to rely on a way to compare the letters to see if they come 'before' or 'after' another letter in order to properly answer the question "does the letter come before the letter N?". But we could have instead asked "is the letter among one of the letters A, B, C, D, E, F, G, H, I, J, K, L, or M?" This may seem like cheating, but this still qualifies as a single yes or no question with a single response.\\

What we are doing is eliminating groups of letters at a time, and we are free to eliminate any number at a time. That is why this game works for a much broader set of objects than just those that can be ordered like letters or numbers. In the game, questions can be things like "is it a person?" or "is it made of metal?".\\

The real reason these kind of questions work is that every object has some defining properties that can be used to differentiate it from all other objects, and the premise of the game is that any property can be reduced to a 'yes' or 'no' question. We can think of each object as either having a property, or not having it, which gives the answer to any possible question. How we differentiate things from each other is the essence of information. If we can't tell any difference, then there is no information.\\

\section{Communication}

\epigraph{...the whole surface of this country would be channelled for those nerves which are to diffuse, with the speed of thought, a knowledge of all that is occurring throughout the land, making, in fact, one neighborhood of the whole country. }{Samuel F. B. Morse}

Information is defined by being able to rule out alternatives. A statement contains no information in itself unless we also know how the statement could have been different. Saying that it is raining outside only contains information because we have alternatives to it raining, such as a clear sky or snowing. The alternatives don't have to necessarily be logical or possible in reality. \\

For instance, we can say that the formula of water is $H_2 O$. This contains information even though we do not think that water actually could have another formula. This is to differentiate the formula of water from some other substance, for example ethanol which is $C_2 H_6 O$. It also does not rule out some other substance having the same formula. For instance diethyl ether is also $C_2 H_6 O$, but it has a different structure which gives it different chemical properties. More information is needed to differentiate chemicals with the same formula.\\

To explore the idea of information we will look at forms of communication: how to send information from one place to another. There are several forms of communication based only on the ability to send what is called a binary signal. The word binary basically means a set of two, or a pair, like in the word bicycle for two wheels. Here a binary signal means a signal that only has two possible states. When we receive one state, the only information we have gained is that we didn't get the other state.\\

Examples of this might be a smoke signal, or a light, or an electrical signal that either on or off. All we can tell is if we can see the signal, or we can't. This is exactly like a 'yes' or 'no'. So, we know we could receive answers to play the 20 questions game by this method, but how could we ask the questions if all we can do is say 'yes' or 'no'?\\

In this case, we seem to need so make some per-arrangements. Suppose ahead of time we worked out a question that had 'yes' or 'no' answer, so that when we are sending the signal they know what question we're answering. This doesn't seem very useful because we would need a way to change which question we're answering for each signal.\\

Since we have the ability to turn the signal on and off over time, we could make longer or shorter signals to mean different things. A short signal could be a 'yes' to one question, and a long signal could be a 'yes' to a different question. Similarly we could think of short or long pauses without a signal as being a 'no' to different questions.

\begin{center}\imagegraphic[0.75]{signals.png}\end{center}

In Morse code, a sequence of short and long signals are used to specify a letter of the alphabet. Short signals (dots) and long signals (dashes) tells us which way to go in the tree of possible letters, and rules out the alternatives, with short pauses in between each signal. A long pause is used to indicate that a new letter is being started.

\index{Morse code}

\begin{center}\imagegraphic[0.75]{morse.png}\end{center}

For example, suppose the first thing we receive is a short signal. In the above tree, you would start at the very top circle, called a node of the tree. If the first signal is a dot (short), then the letter is somewhere in the left side of the tree (E, I, A, S, U, R, W, H, V, F, L, P, or J). If the first signal is a dash, then it must be in the right side of the tree.\\

The next thing that will happen is to have either a short pause or a long pause. If it is a short pause, then we know we need more signals to know what letter it is. If it is a long pause then we have all the signals and we know which letter it is. So, if we receive a short signal and then a long pause, then that means the first letter is an 'E': \mdit.\\

The letter 'X' is in the right hand side of the tree, so the first signal will be a dash. From there it is two dots and then another dash to get to the 'X' branch of the tree: \mdah\mdit\mdit\mdah. Notice that any letter can be specified by as little as 1 signal, and no more than 4 signals. It seems like this requires fewer guesses than in the 20 questions game, but we are getting additional information from the pauses between the signals. We can use the pauses to have different number of signals mean different letters, and so we don't need as many as if we used a fixed number of signals for every letter.

\begin{center}\imagegraphic[0.75]{morse2.png}\end{center}

A word is formed by ending a letter with an even longer pause, and then continuing with the next letter. To form the word cat, we first need a 'c' (\mdah\mdit\mdah\mdit) then an 'a' (\mdit\mdah) then a 't' (\mdah). Putting the letters together with pauses looks like.

\begin{center}
	\mdah\mdit\mdah\mdit\mlet\mdit\mdah\mlet\mdah
\end{center}

Going to the next word is a longer pause. Try decoding the following message: \mdit\mdit\mlet\mdah\mdit\mword
\mdah\mlet\mdit\mlet\mdit\mdah\mlet\mdah\mdit\mdah\mdit\mlet\mdit\mdit\mdit\mdit\mlet\mdit\mdit\mlet\mdah\mdit\mlet\mdah\mdah\mdit\mword \\
\mdah\mdah\mdah\mlet\mdah\mlet\mdit\mdit\mdit\mdit\mlet\mdit\mlet\mdit\mdah\mdit\mlet\mdit\mdit\mdit\mword
\mdit\mdah\mdah\mlet\mdit\mword
\mdah\mlet\mdit\mlet\mdit\mdah\mlet\mdah\mdit\mdah\mdit\mlet\mdit\mdit\mdit\mdit\mword\\
\mdah\mdah\mdah\mlet\mdit\mdit\mdah\mlet\mdit\mdah\mdit\mlet\mdit\mdit\mdit\mlet\mdit\mlet\mdit\mdah\mdit\mdit\mlet\mdit\mdit\mdit\mdah\mlet\mdit\mlet\mdit\mdit\mdit\\

Feel free to come up with your own messages and try sending it to someone over a distance using a two-state signal such as a flashlight.

\section{Digital Signals}

Morse code is an example of a digital signal, but the idea can be generalized to send much more information than letters of the alphabet. Morse code was created at a time when the encoding, transmission, and decoding had to be done manually, and was limited by human speed. Modern technology is able to do these functions much more rapidly, and uses digital signals to send text, audio, video, and other forms of information for cellphones, television, and the internet. So, how does all that information get broken down, if not with dots and dashes?\\

The way this is done is very similar to how sheet music is written. In music, individual notes are supposed to played for a specific amount of time, which is regulated by a beat frequency. Music can be played faster or slower by making the beats shorter or longer, but the relation between how long each note lasts is fixed by the sheet music.\\

In a digital signal, if we know the beat frequency all we need to do is look for \textit{changes} in the signal. Even if the signal does not change we can gain information because it \textit{could} have changed in the same way that every beat \textit{could} be a different note.\\

We already used this kind of scheme with the Morse code by having long versus short signals, and we can be specific about how many beats each element lasts:

\begin{center}
	\begin{tabular}{c | c}
		\textbf{Morse code} & \textbf{\# of beats}\\
		dot & 1 \\
		dash & 3 \\
		pause between elements & 1\\
		pause between letters & 3\\
		pause between words & 7\\
	\end{tabular}
\end{center}

Using these rules we could break up the letter 'C' into 11 individual beats from the code \mdah\mdit\mdah\mdit. In the following image each box represents one beat. If the box is filled in that represents a signal, and if it is empty then there is no signal.

\begin{center}\imagegraphic[0.75]{beatsmorse.png}\end{center}

If we loosen the restrictions on how long each signal or pause has to last, allowing them to last any number of beats, then the total number of different possible sequences is much larger for a given number of beats. If any beat can have either value, then it is called a bit. Bit is short of binary digit. Since this communication is broken down into individual binary digits we call it digital.\\

Each bit has two possible values, and so doubles the number of total possibilities. Two bits have four possible ways they could happen. Three bits have 8 ways. Eleven bits has $2^{11} = 2048$ different ways, and so on.\\

I could set the signals with a regular pattern of bits, perhaps to tell the receiver what the beat frequency is.

\begin{center}\imagegraphic[0.75]{beatsmorse3.png}\end{center}

Or I could set the bits randomly, such as with the flips of a coin.

\begin{center}\imagegraphic[0.75]{beatsmorse2.png}\end{center}

A digital signal is an abstract idea, and so can be implemented by different hardware using different mediums, and converted from one to another. When you load a web page, that stream of bits originated as a file on a hard-drive which was probably stored by magnetic fields on a metal disk. Then converted to electrical pulses when loaded into the servers memory where it was stored by transistors and capacitors.\\

It's then converted again and sent into the network electronically over wires to a service provider. It may then be converted into pulses of light which travel through optical fibers until it reaches another large service provider. Or it is converted to radio waves and transmitted to a satellite which does additional conversion from electrical to radio to send it to another part of the planet. Finally the process is complete bringing it into your computers memory where it is once again stored by little capacitors. Your computer then converts the bits into images and sounds through your browser so that you can experience it on your monitor or speakers.\\

However, we don't have to worry about \textit{how} the information is physically being stored or transmitted. The low level hardware is abstracted away so that we only have to deal with the idea of bits.

\section{Binary Numbers}

\epigraph{There are 10 types of people in the world. Those who know binary, and those who don't.}{\textit{unknown}}

The binary number system can be analyzed in several different ways. In one view, the concept of a particular number exists apart from the way we write it, and decimal versus binary are just two ways of doing it. The number 5 is a concept apart from the symbol '5'. So is the number '6'. But just like with letters, we need a way to tell which number we're talking about. One way to do it is to have a separate symbol for every number, but that is not practical.\\

There is no single way to represent a number, and in fact Morse code has its own way of representing numbers. But we will use a standard representation that closely resembles the decimal system that is useful for doing arithmetic. Instead of basing a digit on a power of ten ($1=10^0, 10=10^0, 100=10^2, 1,000=10^3...$), we base it on a power of 2 ($1=2^0, 2=2^1, 4=2^2, 8=2^3, 16=2^4...$). In binary there are only two symbols: zero (0) and one (1), and represent the binary digits (bits). This is also called base-2, as opposed to base-10, since the 'base' of the exponent used is 2.\\

\begin{center}
\begin{tabular}{c | c | c}
	power & decimal & binary \\
	$2^0$ & 1 & 1 \\
	$2^1$ & 2 & 10 \\
	$2^2$ & 4 & 100 \\
	$2^3$ & 8 & 1000 \\
	$2^4$ & 16 & 10000 \\
	$2^5$ & 32 & 100000 \\
	$2^6$ & 64 & 1000000 \\
\end{tabular}
\end{center}

Like decimal numbers, the least significant digit (the 1's place) starts on the right, and more significant digits go toward the left. To write the number 2, a 1 is put in the 2's place, and a 0 in the 1's place: $2_{(10)} = 10_{(2)}$. I am also using the subscript (10) and (2) to separate base 10 from base 2 numbers. The number 2 looks like 10 in binary because the 1 is in the 2's place, and every zero to the right represents a power of 2, like the zeros represent powers of 10 in the decimal system.\\

Converting a number from binary is fairly straight forward as long as you know the numeric value of each place. If there is a 1 in that place then you add it to the total, and if there is a zero then you don't. For example, the number 1011 has a 1 in the 1s, 2s, and 8s place so we add them together to get $1+2+8=11$. The number 1111 has all four which gives $1+2+4+8 = 15$. And notice that adding them all together is 1 less than the value of the next bit, which is 16.\\

Converting a number to binary starts with the most significant bit that has a value less than the number you want to write in binary. To write the number 37, we start by finding the largest power of two that is less than 37. 64 is too big, so we start with 32. That means the 32s bit is set to one. The remainder is 5 ($37-32$), and so we have to set the next biggest bit that is smaller than 5. 16 and 8 are too big so the next bit is in the 4s place, which leaves a remainder of 1 ($5-4$). 2 is too big so it has to go in the 1s place (which makes sense). Writing the bits out with zeros for the bits that didn't get used gives: 100101. Try converting it back to decimal to verify it gives the correct number.\\

Like with Morse code letters, we can  construct each number based on a yes/no binary method. If we start with the least significant bit, it has two possible values: 0 or 1. Once that bit is determined by moving down one branch of the tree, the second bit has two possible values. Every additional bit has two possible values, doubling the possible number of numbers.\\

\begin{center}\imagegraphic[0.75]{binary.png}\end{center}

You may notice that there are not as many numbers in this tree as there are letters in the Morse code tree. That is because binary numbers are not designed to take advantage of a variable length number of bits. Adding another zero most significant bit does not change the number like adding additional dots did for letters. 0100 is the same number as 100. Converting to/from binary can be thought of as traversing the tree to the number in question. Remember that the top of the tree is the \textit{least} significant bit. So, 1101 leads to the number 13.

\begin{center}\imagegraphic[0.75]{binary2.png}\end{center}

While individual bits carry information, one can see that quite a few bits are required to represent anything useful for us humans. A byte is a short sequence of bits, and is the smallest 'chunk' of information a computer tends deal with at one time. It is currently widely accepted to mean exactly 8 bits, although historically the size of a byte has varied.\\

8 bits have a total number of $2^8=256$ possible values. We can take a continuous stream of bits and group every multiple of 8 together into a continuous stream of bytes. This makes using the stream of information a little easier since each 'chunk' can differentiate between 256 possibilities, instead of just 2.\\

If we are limited to 1 byte, then a binary number has to have exactly 8 digits. The number 0 would be 00000000, the number 1 would be 00000001, and so on. The most significant bit has a value of $2^7 = 128$, and so the byte 10000000 in binary is equivalent to the number 128. If all the bits are set, 11111111 has a value of $2^7 + 2^6 + 2^5 + 2^4 + 2^3 + 2^2 + 2^1 + 2^0 = 255$. So a byte can have integer values from 0 up to 255.\\

Numbers bigger than 255 of course require more bits to be represented. We will deal with how to represent negative numbers and fractions in a later chapter. However, the basic idea is that we are limited by how many possibilities can be represented.

\section{Hexadecimal Numbers}

The name might sound complicated, but it just means base-16. hex (6) + decimal (10) = hexadecimal (16). The idea of using hexadecimal is that binary is just unwieldy to write, and decimal is not a power of 2 which makes converting to and from binary unwieldy as well. Base 16 is a compromise because we can write large numbers more compactly, and also be able to convert to binary very easily.\\

Since this is base 16, each digit can have a value from 0 up to 15, in a similar fashion as base 10 can have a digit from 0 up to 9. But, if we want to only write 1 character per digit, we have a problem because we don't have a single symbol for 15 (or 14, 13, 12, 11, or 10). We need 16 distinct symbols. The solution is to use letters as digits.\\

This might seem strange at first, but in hexadecimal the letter 'a' is actually the number 10. b=11, c=12, d=13, e=14, and f=15. A hexadecimal digit can also be represented by up to 4 bits (or exactly 4 bits if you pad the beginning with zeros).

\begin{center}
	\begin{tabular}{c | c | c}
		decimal & hexadecimal & binary \\
		00 & 0 & 0000 \\
		01 & 1 & 0001 \\
		02 & 2 & 0010 \\
		03 & 3 & 0011 \\
		04 & 4 & 0100 \\
		05 & 5 & 0101 \\
		06 & 6 & 0110 \\
		07 & 7 & 0111 \\
		08 & 8 & 1000 \\
		09 & 9 & 1001 \\
		10 & A & 1010 \\
		11 & B & 1011 \\
		12 & C & 1100 \\
		13 & D & 1101 \\
		14 & E & 1110 \\
		15 & F & 1111 \\
	\end{tabular}
\end{center}

To convert from binary to hexadecimal, you just divide the bits into groups of four, and whatever that number is is the hexadecimal digit for that group. Since a byte is 8 bits, a byte can be written as 2 groups of 4, or two hexadecimal digits: \(11010110_{bin} \rightarrow 1101\quad0110 \rightarrow D6_{hex} \rightarrow 214_{dec}\).

\section{Error Detection}

\epigraph{\imagegraphic[1.0]{parrot.jpg}}{Parity the parrot}

No communication is perfect. There are two levels to dealing with errors. At a minimum we would like to be able to tell \textit{if} there is an error. This is error detection. If there is an error then we know the information can't be trusted, although we might not know what the error is. Doing that requires some amount of redundant information is added to the original message.\\

For example, suppose someone gave you a phone number over the phone and you wrote it down. The extra information might that they repeat the number, and you check to make sure it is the same both times. This will double the amount of information that was transmitted. Now, what can be determined from just a repeated value?\\

Well, there are 2 cases, either the values are the same, or they're different. If they are different, then you would know one of them was wrong, but wouldn't know which one. If they are both the same, then you will probably assume that they are correct. But, it could be that it was wrong both times. The question is, how sure can we be that it is correct? The problem of error detection is that you can never be 100\% sure that the information is correct. You can only be sure if it's not. The idea though is to make the confidence as high as possible.\\

Let's go back to basics for a second. What if we want give a single yes or no response: a single bit of information. Let's assume that from previous experience we know that we make an error about 5\% of the time. If we send an additional bit to error check the first bit, like repeating a phone number, then the only way that we could think that we got a correct response when we didn't is if we made the same error twice in a row. The chances of doing that, based on previous experience, is found by multiplying the chance of 1 error by itself: \(5\% \times 5\% = 0.25\%\). So this is why repeating information increases our confidence in its correctness, since it is now 99.75\% sure instead of only 95\%.\\

Machines tend to be very accurate, and so the chances of a single bit error are much lower than for people. Duplicating the complete information is a little overkill. A method for using a single additional bit to check a whole group of other bits is called a parity check.\\

The way a parity bit works is to make the total number of 1s, including the parity bit, come out to be even. Let's say I sent the message 0110001 (7 bits). Now I will add one more bit, and I will choose that bit such that the total number of 1s is even, so in this case I add a 1 to the end: 0110001\textbf{1}. Let's say the receiver gets one of the bits wrong: 01101011. Well, they add the 1 bits up, and since there's an odd number now they know an error happened. If I sent the message 0011011, I would add a 0 parity bit so that total number of 1s should always be even: 0011011\textbf{0}.\\

Now, say the probability of a single error is one chance in 10,000 (called the bit error rate), and we send 1,000 bits of information. The chance that one of the bits is in error is about 9\%. That's pretty big! But, if we add a parity bit we can detect if that happens, which it will about 9\% of the time, and ask for the information again. However, a parity bit won't work if there are two errors (or any even number of errors). Suppose I send the message 0110111, add parity bit 1: 0110111\textbf{1}. But two errors happen on the other end: 01111101. They see an even number of 1s, so they don't know an error happened.\\

The chances that there are two errors is 0.4\%, so 0.4\% of the time an error will happen but go undetected. We can 'dial' in how sure we want to be, given how unsure we are about the transmission channel, by adding parity bits more frequently. So, if we put a parity bit every 8 bits, then the chances of an error going undetected would be 0.00003\%. (this is all assuming a bit error rate of 1 in 10,000)

\section{Error Correction}

Another level on top of error detection is if we could determine \textit{what} the error is. With a simple parity bit check, we can only detect that an error occurred, but not which bit it was. If we know what the error is, we can change it to the correct value without having to ask for it again.\\

Suppose we start out sending a stream of data putting a parity bit every 8 bits to do error checking. We notice there is an error in one of the groups. How can we figure out which bit in that group caused the problem? If we think about this in two dimensions, it's a little easier to see how this might work. I can re-write the following hexadecimal number as a series of bits arranged in a grid, with each row being 8 bits (or 2 digits) of the original number.\\

\begin{center}6C C9 44 4D 33 A0 03 3C\end{center}

\begin{center}\imagegraphic[0.75]{errorcorrect.png}\end{center}

Each row represents one group of 7 message bits plus 1 parity bit on the right. Each row should have an even number of 1s. If not, then we know there is an error somewhere on that row. There are 7 rows of message bits, plus a new row entirely of parity bits, where each of those bits correspond to the columns. The number of 1s in a column should also add up to an even number. If a single bit error occurs anywhere, then we can determine which row it's on, and which column. Knowing the row and column will tell us exactly which bit went wrong, and we can fix it.\\

Suppose the receiver of the above bits has the following result. They would go to each row and see if the 1s add up to an even number. If not, they mark that row as having an error. Then go each column, and mark any column with an odd number of 1s. Whatever bit ends up at that row and column should have the opposite value.

\begin{center}6C C9 \textbf{6}4 4D 33 A0 03 3C\end{center}

\begin{center}\imagegraphic[0.75]{errorcorrect2.png}\end{center}


\section{Exercises}

\begin{enumerate}
	\item A single flashlight has two possible states that can be perceived by a person at a distance (on and off). How many different states could be perceived using two flashlights? Assume the flashlights are identical. If they are held side-by-side, could someone tell which one was on, when only one of them is on? If someone is far enough away, will the two flashlights appear as a single flashlight when they are both on?
	
	\item Suppose a red filter is added to one of the flashlights, and a blue filter is added to the other one. How will having two different colors change how many possible states can be perceived by a person at a distance away? If the two lights look like a single light from far away, what color will they appear when both flashlights are on?
	
	\item Suppose we have a 4 state transmission system labeled by the following digits: 0 (off), 1 (on), 2 (on), 3 (on). Come up with a code that uses the off state to separate signals, the letters of the alphabet are specified by a sequence of digits, and a space is also a sequence of digits. What is the minimum sequence of digits needed to send any of the 26 letters or a space? 
\end{enumerate}